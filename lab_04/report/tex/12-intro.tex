\StructuredChapter{Введение}

Ускорителями вычислений принято называть специальные аппаратные устройства, способные выполнять ограниченный ряд задач с большей параллельностью и за меньшее время в сравнении с универсальными микропроцессорными ЭВМ . Как правило, ускоритель представляет собой структуру, включающую большое количество примитивных микропроцессорных устройств, объединенных шинами связей.

В настоящее время применение ускорителей вычислений охватывает ряд важных областей: финансовые вычисления, ускорение запросов к базам данных, машинное обучение, видео-аналитика. В ряде случаев удается достичь ускорения более чем в 90 раз по сравнению с универсальными ЭВМ, построенными на микропроцессорах Intel x86.

\textbf{Цель работы}: изучение архитектуры гетерогенных вычислительных систем и технологии разработки ускорителей вычислений на базе ПЛИС фирмы Xilinx.
Для достижения данной цели необходимо выполнить следующие задачи:
\begin{itemize}
\item изучить основные сведения о платформе Xilinx Alveo U200;
\item разработать RTL описание ускорителя вычислений по индивидуальному варианту;
\item выполнить генерацию ядра ускорителя;
\item выполнить синтез и сборку бинарного модуля ускорителя;
\item разработать и отладить тестирующее программное обеспечение на серверной хост-платформе;
\item провести тесты работы ускорителя вычислений.
\end{itemize}

Все задания выполняются в соответствии с вариантом №4.

